%-------------------------
% Rezume, a latex resume template for developers
% Author : Nanu Panchamurthy
% Based off of: https://github.com/sb2nov/resume
% License : MIT

% Hope this resume template helps you land an awesome job. If you found this helpful, please consider starring the github repo here, .
%-------------------------



%------------PACKAGES----------------
\documentclass[a4paper,11pt]{article}

\usepackage{verbatim} % reimplements the "verbatim" and "verbatim*" environments

\usepackage{titlesec} % provides an interface to sectioning commands i.e. custom elements

\usepackage{color} % provides both foreground and background color management

\usepackage{enumitem} % provides control over enumerate, itemize and description

\usepackage{fancyhdr} % provides extensive facilities for constructing headers, footers and also controlling their use

\usepackage{tabularx} % defines an environment tabularx, extension of "tabular" with an extra designator x, paragraph like column whose width automatically expands to fill the width of the environment

\usepackage{latexsym} % provides mathematical symbols

\usepackage{marvosym} % provides martin vogel's symbol font which contains various symbols

\usepackage[empty]{fullpage} % sets margins to one inch and removes headers, footers etc..

\usepackage[hidelinks]{hyperref} % removes color and shadow of hyperlinks

\usepackage[normalem]{ulem} % provides "\ul" (uline) command which will break at line breaks

\usepackage[english]{babel} % provides culturally determined typographical rules for wide range of languages
%-----------------------------------------

\input glyphtounicode % converts glyph names to unicode
\pdfgentounicode=1 % ensures pdfs generated are ats readable

%----------FONT OPTIONS-------------------
\usepackage[default]{sourcesanspro} % uses the font source sans pro
\urlstyle{same} % changes url font from default urlfont to font being used by the document
%-----------------------------------------


%----------MARGIN OPTIONS-----------------
\pagestyle{fancy} % set page style to one configured by fancyhdr
\fancyhf{} % clear all header and footer fields

\renewcommand{\headrulewidth}{0in} % sets thickness of linerule under header to zero
\renewcommand{\footrulewidth}{0in} % sets thickness of linerule over footer to zero

\setlength{\tabcolsep}{0in} % sets thickness of column separator in tables to zero

% origin of the document is one inch from the top and from and the left
% oddsidemargin and evensidemargin both refer to the left margin
% right margin is indirectly set using oddsidemargin
\addtolength{\oddsidemargin}{-0.5in}
\addtolength{\topmargin}{-0.5in}

\addtolength{\textwidth}{1.0in} % sets width of text area in the page to one inch
\addtolength{\textheight}{1.0in} % sets height of text area in the page to one inch

\raggedbottom{} % makes all pages the height of current page, no extra vertical space added
\raggedright{} % makes all pages the width of current page, no extra horizontal space added
%------------------------------------------


%--------SECTIONING COMMANDS---------
% \titleformat{<command>}
%   [<shape>]{<format>}{<label>}{<sep>}
%     {<before-code>}[<after-code>]

% command is the sectioning command to be redefined
% shape is the style of the font; scshape stands for small caps style
% format is the format to be applied to whole title- label and text; absent here
% label defines the label
% sep is the horizontal separation between label and title body
% before-code is the code to be executed before
% after-code is the code to be executed after

\titleformat{\section}
  {\scshape\large}{}
    {0em}{\color{blue}}[\color{black}\titlerule\vspace{0pt}]
%-------------------------------------


%--------REDEFINITIONS----------------
% redefines the style of the bullet point
\renewcommand\labelitemii{$\vcenter{\hbox{\tiny$\bullet$}}$}

% redefines the underline depth to 2pt
\renewcommand{\ULdepth}{2pt}
%-------------------------------------


%--------CUSTOM COMMANDS--------------
%\vspace{} defines a vertical space of given size, modifying this in custom commands can help stretch or shrink resume to remove or add content

% resumeItem renders a bullet point
\newcommand{\resumeItem}[1]{
  \item\small{#1}
}

% commands to start and end itemization of resumeItem, rightmargin set to 0.11in to avoid the overflow of resumetItem beyond whatever resumeItemHeading is being used
\newcommand{\resumeItemListStart}{\begin{itemize}[label={$\bullet$},rightmargin=0.11in]}
\newcommand{\resumeItemListEnd}{\end{itemize}}

% resumeSectionType renders a bolded type to be used under a section, used as skill type here, middle element is used to keep ":"s in the same vertical line
\newcommand{\resumeSectionType}[3]{
  \item\begin{tabular*}{0.96\textwidth}[t]{
    p{0.15\linewidth}p{0.02\linewidth}p{0.81\linewidth}
  }
    \textbf{#1} & #2 & #3
  \end{tabular*}\vspace{-2pt}
}

% resumeTrioHeading renders three elements in three columns with second element being italicized and first element bolded, can be used for projects with three elements
\newcommand{\resumeTrioHeading}[3]{
  \item\small{
    \begin{tabular*}{0.96\textwidth}[t]{
      l@{\extracolsep{\fill}}c@{\extracolsep{\fill}}r
    }
      \textbf{#1} & \textit{#2} & #3
    \end{tabular*}
  }
}

% resumeQuadHeading renders four elements in a two columns with the second row being italicized and first element of first row bolded, can be used for experience and projects with four elements
\newcommand{\resumeQuadHeading}[4]{
  \item
  \begin{tabular*}{0.96\textwidth}[t]{l@{\extracolsep{\fill}}r}
    \textbf{#1} & #2 \\
    \textit{\small#3} & \textit{\small #4} \\
  \end{tabular*}
}

% resumeQuadHeadingChild renders the second row of resumeQuadHeading, can be used for experience if different roles in the same company need to added
\newcommand{\resumeQuadHeadingChild}[2]{
  \item
  \begin{tabular*}{0.96\textwidth}[t]{l@{\extracolsep{\fill}}r}
    \textbf{\small#1} & {\small#2} \\
  \end{tabular*}
}

% commands to start and end itemization of resumeQuadHeading, lefmargin for left indent of 0.15in for resumeItems
\newcommand{\resumeHeadingListStart}{
  \begin{itemize}[leftmargin=0.15in, label={}]
}
\newcommand{\resumeHeadingListEnd}{\end{itemize}}
%-------------------------------------------


%__________________RESUME____________________
% You can rearrange sections in any order you may prefer
\begin{document}

%-----------CONTACT DETAILS------------------
% Make sure all the details are correct, you can add more links in the first row of second column if needed

\begin{tabular*}{\textwidth}{l@{\extracolsep{\fill}}r}
  \textbf{\Huge{\href{https://ankitb.dev}{Ankit Singh}} \vspace{2pt}} & % row = 1, col = 1
   \\ % row = 1, col = 2
  \href{https://ankitb.dev}{\uline{Portfolio}} $|$ % row = 2, col = 1
  \href{https://www.linkedin.com/in/ankit-singh-58304221b/}{\uline{LinkedIn}} $|$ % row = 2, col = 1
  \href{https://github.com/0205Ankit}{\uline{GitHub}} $|$ % row = 2, col = 1
  Email: \href{mailto:singhankit8066@gmail.com}{\uline{singhankit8066@gmail.com}} $|$ % row = 2, col = 2
  Mobile: +91-9119018066 \\ % row = 2, col = 2
\end{tabular*}
%--------------------------------------------


%-----------SUMMARY--------------------------
% Keep this short, simple and straigth to point

% \section{Full Stack Developer}
% \small{
% Full Stack Developer with hands-on experience crafting responsive web applications using \textbf{JavaScript}, \textbf{Node.js}, and \textbf{React.js}. Adept at building scalable solutions and integrating \textbf{RESTful APIs}. Excited about leveraging technology to solve real-world problems and continuously learning to stay at the forefront of the industry.
% }
%--------------------------------------------


%--------------SKILLS------------------------
% Add or remove resumeSectionTypes according to your needs

% \section{Technical Skills}
%   \resumeHeadingListStart{}
%     \resumeSectionType{Languages}{:}{JavaScript, TypeScript, C++, HTML, CSS}
%     \resumeSectionType{Frameworks}{:}{React.js,  Express, Node.js, Next.js}
%     \resumeSectionType{Libraries}{:}{Radix, Redux/Redux toolkit, TRPC, Prisma}
%     \resumeSectionType{Databases}{:}{MongoDB, PostgreSQL}
%     \resumeSectionType{Dev Tools}{:}{Visual Studio Code, Git, Github}
%   \resumeHeadingListEnd{}
%--------------------------------------------


%-----------EXPERIENCE-----------------------
% Distill all your talking points to small bullet points which follow the pattern "challenge-action-result" for maximum efficiency. Try to quantify (use numbers) your points whenver possible, highlist words of importance

\section{Professional Experience}
\resumeHeadingListStart{}
  \resumeQuadHeading{Sofware Developer Intern}{Oct 2023 -- Jan 2024}
  {FalconAI}{Bangalore, Karnataka, India}
    \resumeItemListStart{}
      \resumeItem{Led a major overhaul of service workflows, optimizing efficiency and user experience using \textbf{Next.js} and \textbf{Prisma}.}
      \resumeItem{Engineered cutting-edge AI solutions in collaboration with \textbf{LangChain}, and \textbf{OpenAI} API.}
      \resumeItem{\textbf{Optimized} data processing workflows, resulting in a \textbf{30\% reduction} in data processing time and a \textbf{20\% increase} in overall system efficiency}
      \resumeItem{Authored \textbf{scalable}, \textbf{maintainable} code within the Next.js and Prisma framework, preparing for future growth.}
      % \resumeItem{Pursued continuous professional development, swiftly adapting to emerging technologies to remain industry-leading.}
    \resumeItemListEnd{}
    
\resumeQuadHeading{React Developer Intern}{Feb 2023 -- Apr 2023}
  {Itaxeasy}{Noida, Uttar Pradesh, India}
    \resumeItemListStart{}
      \resumeItem{Developed dynamic, responsive websites using \textbf{HTML, CSS, React, and Express}, enhancing user interaction and experience.}
      \resumeItem{Utilized \textbf{REST APIs} for efficient data retrieval and display, enhancing user experience.}
      \resumeItem{\textbf{Optimized} website loading performance by \textbf{40\%}, significantly boosting speed and responsiveness.}
    \resumeItemListEnd{}
\resumeHeadingListEnd

%---------------------------------------------


%-----------EDUCATION-------------------------
% Mention your CGPA, if its good, in the first row of second column

\section{Education}
  \resumeHeadingListStart{
    \resumeQuadHeading{B.Tech , Meerut Institute Of Engineering and Technology}{Meerut, Uttar Pradesh}
    {(Electrical Engineering) 73\%}
    {Jun 2019 -- Aug 2023}
    {Affiliated with (A.K.T.U) Abdul Kalam Technical University}
  \resumeHeadingListEnd
%---------------------------------------------


%-----------PROJECTS--------------------------
% Use resumeQuadHeading if four elements are feasible (ex: demo video link), else use resumeTrioHeading. Keep the bullet points simple and concise and try to cover wide variety of skills you have used to build these projects

\section{Projects}
  \resumeHeadingListStart{}
  \resumeTrioHeading{\href{https://app.falconai.in}{\uline{FalconAI}}}{Next.js, Prisma, Supabase, Typescript, TRPC, Git}{\href{https://app.falconai.in}{\uline{Link}}}
      \resumeItemListStart{}
        \resumeItem{Spearheaded the development of FalconAI, a platform aimed at improving teacher-student communication through \textbf{AI-powered tools}.}
        \resumeItem{Enabled teachers to generate tasks, lesson plans, and materials, while the AI \textbf{autonomously} monitored and \textbf{conducted tasks} on the student end.}
        \resumeItem{Implemented a \textbf{chat interface} for interactive task execution, fostering engagement and \textbf{personalized learning} experiences.}
        \resumeItem{Create \textbf{chatbot} with your \textbf{personal} persona}
      \resumeItemListEnd

    
    \resumeTrioHeading{\href{https://chat-frame-coral.vercel.app/}{\uline{ChatFrame}}}{Next.js, Express.js, Typescript, TRPC, Socket.io,  Supabase, TailwindCSS}{\href{https://chat-frame-coral.vercel.app/
}{\uline{Link}}}
      \resumeItemListStart{}
        \resumeItem{Designed and built ChatFrame, a \textbf{real-time chat} application with seamless \textbf{photo-sharing} and \textbf{robust messaging} functionalities.}
        \resumeItem{Integrated features for both \textbf{one-to-one} and \textbf{group}, messaging including support for multimedia content such as \textbf{images} and \textbf{audio}.}
        \resumeItem{Implemented \textbf{real-time, in-app notification} systems to ensure prompt updates for incoming messages, enhancing user engagement.}
        \resumeItem{\textbf{Additionally}, the project encompasses all fundamental functionalities of a \textbf{social media} application.}
      \resumeItemListEnd{}

      
  \resumeHeadingListEnd{}

\section{Technical Skills}
  \resumeHeadingListStart{}
    \resumeSectionType{Languages}{:}{JavaScript, TypeScript, C++, HTML, CSS}
    \resumeSectionType{Tools}{:}{React.js, Node.js, Next.js, TailwindCSS, Docker, Supabase, Redux,Socket.io}
    \resumeSectionType{Databases}{:}{MongoDB, PostgreSQL}
    \resumeSectionType{Dev Tools}{:}{Visual Studio Code, Git, Github}
  \resumeHeadingListEnd{}
%--------------------------------------------


%----------------OTHERS----------------------
% You can add your acheivements, accolades, certifications etc. here.

%--------------------------------------------

\end{document}
